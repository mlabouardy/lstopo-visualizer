\documentclass[12pt]{article}

\usepackage{enumerate} % allows us to customize our lists
\usepackage[utf8]{inputenc}

\begin{document}

\title{Rapport du Projet d'Etude et de Développement}
\author{GreenScrum}
\date{\today}

\maketitle

\section{Introduction}
\subsection{Présentation du projet}
\paragraph{}
	Le but du projet est de proposer un logiciel de visualisation au format HTML d'une topologie de machines. Cette topologie est constituée d'informations matérielles de chaque machine. Ces données servent à àméliorer les calculs parallèles en optimisant l'utilisation du matériel. Les données à utiliser sont extraites du module lstopo de la bibliothèque logicielle hwloc

\subsection{Réalisation du projet}
Le projet devait être fait en 2 parties avec une partie en C et une partie Web. Le but de la 1ère partie devait être de créer une exportation des données à visualiser. Mais, après discussion avec le client, nous avons compris que ce n'est pas nécessaire. En effet, une exportation au format XML existe déjà, et elle est suffisante pour fournir les données nécessaires à la création de l'application de visualisation.

\end{document}
