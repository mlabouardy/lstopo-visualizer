\documentclass [a4paper,11pt]{article}

\usepackage [francais]{babel}
\usepackage[utf8]{inputenc}%       encodage du fichier source
\usepackage[T1]{fontenc}%          gestion des accents (pour les pdf)
\usepackage[a4paper]{geometry}%    taille correcte du papier
\usepackage{hyperref}%			   gestion des hyperliens
\usepackage{graphicx}%			   gestion des images
\usepackage{fancyhdr}%
\usepackage{lastpage}%			   pour avoir le numero de page
\usepackage{fancyvrb}%			   meilleur verbatim
\usepackage{chngcntr}%			   pour numéroté les figures au lieux du numéro de section
\usepackage[toc,page]{appendix}%   pour les annexes

% Chemin par defaut des images
\graphicspath{{img/}}

\pagestyle{fancyplain}
\fancyhf{}
\cfoot{\thepage\ sur \pageref{LastPage}}


\begin{document}

\begin{titlepage}
\begin{center}
{\bf Université Sciences et Technologies - Bordeaux1} \vspace{0.5cm}\\

{\bf {\large Master 2 Informatique : Genie logiciel parcours conduite de projet}}\\
%{\emph{Rapport du Projet d'Etude et de Développement }}\\\vspace{1cm}

\begin{figure}[!ht]
  \centering
  \includegraphics[scale=0.2]{img/uniBx-logo}

  \label{fig:logUniBx}
\end{figure}


{\large{\bf{Rapport:}}}\\\vspace{1cm}
{\huge{\bf Visualisation interactive de topologie de plates-formes parallèles avec lstopo et HTML}}\\\vspace{0.5cm}





\end{center}

\hspace{1cm}\textbf{Réalisé par:} GreenScrum \\
%\bigskip

\hspace{1cm}\textbf{Encadré par:} Philippe Narbel , David Auber.\\


\hspace{1cm}\textbf{Client:} Brice Goglin\\



\end{titlepage}





\tableofcontents

\newpage

\section{Introduction}

\subsection{Présentation du projet}
Le but du projet est de proposer un logiciel de visualisation au format HTML d'une topologie de machines. Cette topologie est constituée d'informations matérielles de chaque machine. Ces données servent à àméliorer les calculs parallèles en optimisant l'utilisation du matériel. Les données à utiliser sont extraites du module lstopo de la bibliothèque logicielle hwloc

\subsection{Réalisation du projet}
Le projet devait être fait en 2 parties avec une partie en C et une partie Web. Le but de la 1ère partie devait être de créer une exportation des données à visualiser. Mais, après discussion avec le client, nous avons compris que ce n'est pas nécessaire. En effet, une exportation au format XML existe déjà, et elle est suffisante pour fournir les données nécessaires à la création de l'application de visualisation.

\newpage

\section{Organisation du projet}

\subsection{Gestion de projet}

\subsection{Technologies utilisées}

\subsection{Tests}

\subsubsection{Intégration Continue}

Intégration continue est une étape importante à mettre en place dans le processus de développement logiciel. Nous parlerons donc ici de cette pratique et de l'outil utilisé au cours de ce projet.

\begin{table}[h]
\begin{center}
\includegraphics[scale=0.5]{img/ci.png}
    \caption{Intégration continue workflow}
\end{center}
\end{table}

\newpage

\section{Développement}

\subsection{Architecture}

\subsection{Problèmes rencontrés}

\subsection{Résultat}

\newpage

\section{Critique}

\subsection{Mise en oeuvre du projet}

\subsection{Outils utilisés}

\subsection{Améliorations possibles}

\newpage

\nocite{*}
\bibliographystyle{plain}
\bibliography{report}




\end{document}


